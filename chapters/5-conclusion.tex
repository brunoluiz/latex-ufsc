\chapter{Conclusão}
\label{stateofart:conclusion}

  O objetivo de estudar o funcionamento de uma \textit{\ac{EVSE}} e como implementá-la foi alcançado com sucesso após os testes finais na estação protótipo. Embora essa primeira entrega não permita a utilização do conector Combo, já é possível realizar carregamentos em bicicletas e carros (modo lento), o que permite a utilização imediata da estação.

  % FIXME: Falar um pouco mais do que aprendeu durante o projeto

  A complexidade do \textit{software}, assim como todas tecnologias utilizadas nele, permitiram que os objetivos relacionados à \textit{software} fossem alcançados também. Embora houveram problemas ao transferir a aplicação da bancada de testes para a estação, estes permitiram o aprendizado de métodos para resolução de problemas.

  Cabe um agradecimento à Fundação CERTI e CELESC, pois esse projeto acadêmico foi possibilitado por estas. O projeto, como um todo, faz parte do programa P\&D ANEEL, que busca promover a cultura da inovação no setor elétrico brasileiro.
  