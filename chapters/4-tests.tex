\chapter{Testes e Resultados}
\label{tests:tests}

  \section{Ferramentas de teste}
  \label{tests:tools}

    Para testar a implementação realizada, é necessária uma ferramenta para simular um servidor central. Inicialmente, testou-se a ferramenta \href{http://www.gir.fr/ocppjs/}{GIR OCPPJS}, porém essa apresentou problemas ao lidar com algumas das requisições vindas da \ac{EVSE}.

    Foi criada então uma ferramenta para testes, disponível em \url{https://github.com/brnluiz/ocpp-tools}, que implementa parcialmente as funções de um sistema central. Todas operações apresentadas na tabela \ref{table:ocpp} foram implementadas.

    Após aberta, a ferramenta ouve todas requisições das estações que conectadas e responde as requisições com dados genéricos, o que possibilita o funcionamento da estação protótipo. Caso o desenvolvedor precise executar alguma requisição às estações, é possível utilizar a \textit{\ac{CLI}} para tal (os comandos disponíveis são dados pelo manual da ferramenta).

  \section{Implementações e testes iniciais}
  \label{tests:initial}

    O desenvolvimento inicial se deu no estudo da base de código anterior, que ainda não implementava nada da \ac{EVSE} funcionalmente, apenas possibilitava o teste de cada dispositivo isoladamente. Após esse período inicial, iniciou-se a implementação do software da estação.

    Alguns testes unitários foram desenvolvidos sob o \textit{framework} de testes \textit{JUnit}, o que permite o programa ser testado antes de ser embarcado. Assim que todos testes passam, o programa é enviado para a \textit{BeagleBone}, a qual está conectada aos dispositivos listados na seção \ref{methodology:devices}, porém dispostos em uma bancada de testes (figura \ref{fig:setup-tests}).

    Como as cargas utilizadas durante o teste consumiam pouca energia (fontes \ac{CA}-\ac{CC} de dispositivos de baixo consumo, como notebooks), o medidor de energia teve seu multiplicador de corrente configurado em 100, possibilitando assim simular um consumo alto de forma rápida (sem precisar aguardar um grande período de tempo para atingir 1 kWh).

    Simulou-se a inicialização e finalização de cargas no conector modo 2, onde foram seguidos fluxos similares as figuras \ref{fig:sw-starttransaction} e \ref{fig:sw-stoptransaction}, já que essas permitem checar diversos aspectos do sistema: autenticação, comunicação externa e com medidores, criação de transações, checagem de falhas e entre outras.

    O conector modo 3 Mennekes não pode ser testado na bancada de testes pois, para habilitar o carregamento, é necessário realizar todo sequenciamento apresentado na figura \ref{fig:phoenix} e este só seria possível se houvesse um veículo próximo a bancada. Portanto, este ficou para ser testado na própria estação protótipo.

    Todos arquivos, sistema operacional incluso, são colocados em um cartão SD, o que facilita a cópia desses dados para backup (criação de imagens) e permite o intercâmbio do sistema entre a bancada de testes e a estação protótipo.

    \begin{figure}[H]
      \begin{center}
        \includegraphics[width=0.6\textwidth,natwidth=2130,natheight=1420,angle=-90]{assets/images/setup-tests.jpg}
        \caption{Disposição dos dispositivos na bancada de testes}
        \label{fig:setup-tests}
      \end{center}
    \end{figure}

  \section{Testes na estação protótipo}
  \label{tests:evse}

    Após executar os testes na bancada, o sistema foi levado para a \textit{\ac{EVSE}} protótipo, situada no estacionamento da Fundação CERTI. A disposição dos dispositivos da estação difere com a da bancada de testes (figura \ref{fig:setup-evse}), porém visto que os dispositivos são os mesmos e o sistema embarcado foi colocado em um cartão SD, só seria necessário inserir o cartão na BeagleBone da estação e configurá-la para dar boot pelo cartão, caso já não estiver configurada de fábrica assim.

    \begin{figure}[H]
      \begin{center}
        \includegraphics[width=\textwidth,natwidth=1420,natheight=2130]{assets/images/setup-evse.jpg}
        \caption{Disposição dos dispositivos na estação protótipo}
        \label{fig:setup-evse}
      \end{center}
    \end{figure}

    Embora a previsão fosse de que a estação funcionaria igualmente a bancada de testes, ao menos para o conector modo 2, não foi isso que ocorreu inicialmente. Um dos problemas que ocorreu foi relativo a conectividade da BeagleBone com outros dispositivos Ethernet, que foi resolvido com a substituição do \textit{connman} pelo \textit{Network Manager}. O \textit{connman} possui configurações atreladas a dados únicos de cada BeagleBoard, o que iria requerer que cada BeagleBone fosse configurada manualmente. Enquanto isso, o \textit{Network Manager} se baseia em configurações fixas no arquivo /etc/network/interfaces, sendo que esse não causou problemas quando executado na estação.

    Os outros foram relacionados a serial: um foi devido as configurações seriais da bancada e da estação serem diferentes, porém isso foi rapidamente detectado e resolvido. Porém, um problema um pouco mais complicado surgiu logo após, quando notamos que o \textit{software} não estava funcionando adequadamente na estação pois não conectava ou perdia leituras. Foi descoberto que alguns circuitos integrados que transformam a saída \textit{\ac{UART}} para RS-485 possuem sua saída conectada a sua entrada, o que acabava gerando problemas na recepção de alguns dados. Após alguns dias foi descoberto que isso se chamava \textit{echo} e poderia ser resolvido via \textit{software} (parâmetro \textit{"echo"} da classe de conexão do Jamod).

    Além disso, surgiram problemas no conector modo 3 Mennekes. O conector do veículo de testes, Mitsubishi Miev, é do padrão americano. Para carregá-lo é necessário de um adaptador para o padrão Mennekes. Porém, tal adaptador acabou gerando falhas elétricas que impossibilitem o carregamento contínuo do veículo, como mal contato entre os pinos do adaptador com o conector. No momento em que alguma falha é percebida pelo \textit{Phoenix}, este desativa o carregamento imediatamente. Para resolver, o adaptador foi cortado em alguns milímetros para possibilitar que os pinos de contato ficassem mais próximos, além de o \textit{software} mandar um comando para o \textit{Phoenix} limitar a corrente máxima permitida (visto que há suspeita de que, devido ao adaptador, o carregamento não pode exigir muita potência).

    Após resolvidos todos problemas, a \textit{\ac{EVSE}} funcionou de maneira esperada e com ambos conectores habilitados.

    Em um teste com um BMW i3, o carregamento não apresentou nenhum problemas e foi possível uma carga completa utilizando o conector Mennekes. A figura \ref{fig:evse-charge} mostra tal carregamento, onde a estação chega a fornecer um máximo de 24.5 A e leva 4h30m para realizar o carregamento. O carregamento poderia ser ainda mais rápido caso fosse utilizado um conector modo 4 (seção \ref{stateofart:modes:europe:mode4}).

    \begin{figure}[H]
      \begin{center}
        \includegraphics[width=\textwidth,natwidth=1420,natheight=2130]{assets/images/setup-evse.jpg}
        \caption{Gráfico Corrente x Tempo para carregamento do BMW i3 utilizando o conector modo 3 Mennekes}
        \label{fig:evse-charge}
      \end{center}
    \end{figure}

    O programa gera saídas de texto para diversas operações que realiza internamente, o que permite a visualização do que está ocorrendo na estação por parte do desenvolvedor. Para fins de demonstração, foi realizada uma carga de menos de um minuto apenas para mostrar o funcionamento do sequenciamento da estação. A saída gerada está disponível no apêndice \ref{appendix:log} e é possível observar o \textit{software} realizando algumas das etapas mostradas nos sequenciamentos das figuras \ref{fig:sw-init}, \ref{fig:sw-starttransaction} e \ref{fig:sw-stoptransaction}.

    \begin{itemize}
      \item Linha 34: usuário utiliza a \textit{\ac{RFID}}
      \item Linha 35: \textit{\ac{RFID}} é autenticada (usuário habilitado para carregamento)
      \item Linhas 39-46: \textit{software} aguarda usuário selecionar o conector
      \item Linhas 41-45: inicia a \textit{Thread} da transação e habilita o carregamento
      \item Linhas 48-54: notifica servidor \textit{\ac{OCPP}} sobre o início do carregamento
      \item Linha 46: mostra mensagem para o usuário sobre o início do carregamento
      \item Linhas 55-86: \textit{software} envia dados para servidor e alterna entre algumas telas na \ac{IHM}
      \item Linhas 87-88: carregamento cancelado devido desconexão do veículo por parte do usuário
      \item Linhas 89-95: notifica servidor \textit{\ac{OCPP}} sobre o término do carregamento
      \item Linhas 96-99: aguarda novas iterações do usuário
    \end{itemize}