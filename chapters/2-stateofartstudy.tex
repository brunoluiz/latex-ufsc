\chapter{Revisão Bibliográfica}
\label{stateofart}

  \section{Mercado de Veículos Elétricos}
  \label{stateofart:market}

    O mercado de veículos elétricos, desde o início da década, está aquecido. Nos Estados Unidos, de 2010 à 2014 houve um aumento de 71\% no volume de vendas (286.390 mil veículos em números absolutos), como pode ser visto na tabela \ref{table:american-sales} \cite{fsec-report-ev}. Na China, em 2015, houve um aumento de 4,2 vezes na produção de carros movidos a bateria, e 3,3 vezes quando incluir outras categorias de \textit{\ac{EV}} \cite{caam-report-ev}.

    % Please add the following required packages to your document preamble:
    % \usepackage{booktabs}
    \begin{table}[]
    \centering
    \caption{Vendas nos Estados Unidos}
    \label{table:american-sales}
    \begin{tabular}{@{}lllll@{}}
    \toprule
    \textbf{} & \textbf{Híbridos} & \textbf{Somente bateria} & \textbf{Anual} & \textbf{Cumulativo} \\ \midrule
    2010             & 326                       & 19                      & 345             & 345                  \\
    2011             & 7,671                     & 10,064                  & 17,735          & 18,080                \\
    2012             & 38,584                    & 14,251                  & 52,835          & 70,915                \\
    2013             & 49,008                    & 47,694                  & 96,702          & 167,617               \\
    2014             & 55,357                    & 63,416                  & 118,773         & 286,390               \\ \bottomrule
    \end{tabular}
    \end{table}

    Já na União Europeia, de 2013 para 2014 houve um crescimento de 31,5\%, com a venda de 22,6 mil veículos. Embora o crescimento aparente ser grande, está longe de ser maioria no mercado: menos de 1\% da atual frota é constituída de veículos elétricos e a adoção destes depende muito dos incentivos dados em cada país. No contexto Europeu, porém fora da União Europeia, a Noruega se destaca, com uma fatia de 22,5 \% de veículos elétricos nas vendas de 2015. \cite{eaa-report-ev}

    Embora esse mercado apresente potencial, ainda há uma grande dependência em fatores externos. Exemplo disso são os atuais fabricantes possuírem poucos modelos, sendo em alguns casos somente uma adaptação de um modelo movido a combustão, pois ainda estão testando o mercado e avaliando seus investimentos.

    Alguns fatores influenciam a adoção de veículos elétricos, como:

    \begin{itemize}
      \item Regulamentações exigindo os fabricantes automotores a reduzir emissões de CO$_2$
      \item Incentivos fiscais e financeiros para a aquisição de veículos e operações relacionadas
      \item Preços de combustíveis
      \item Diminuição de custos de baterias e evolução da tecnologia
      \item Infraestrutura do ecossistema de veículos elétricos (pontos de carregamento e manutenção)
    \end{itemize}

    Alguns países Europeus, como a Alemanha, já afirmaram que desejam abolir vendas de veículos a combustão \cite{forbes-news-germany}. Outros países planejam seguir a mesma linha, o que força os fabricantes a iniciarem pesquisas na área e desenvolver as tecnologias relacionadas a esse mercado.

  \section{Modos de Carregamento}
  \label{stateofart:modes}

    \subsection{Modos Europeus}
    \label{stateofart:modes:europe}

      Os modos de carregamento europeus seguem o padrão IEC 62196 \cite{iec-62196}, que apresenta quatro modos de operação.

        \subsubsection{Modo 1 - Carregamento Lento}
        \label{stateofart:modes:europe:mode1}

        Utiliza o conector de tomada residenciais, o qual não pode exceder 16 A de corrente e 250 V \ac{CA} monofásica ou 480 V \ac{CA} de tensão trifásica. Normalmente demanda de 6 à 8 horas para carregar completamente o veículo, o que é considerado um carregamento lento. Como é utilizada uma tomada comum, não é necessária uma \textit{\ac{EVSE}} para fazer o controle desse carregamento, mas somente uma extensão que conecta a tomada ao veículo. Porém, como esse padrão utiliza uma proteção diferencial e requer uma proteção aterrada na residência, muitos países não o adotam, visto que boa parte das residências não possuem aterramento. Nesse modo, o veículo precisa possuir com um inversor interno para a conversão \ac{CA}-\ac{CC}.

        \subsubsection{Modo 2 - Carregamento Lento}
        \label{stateofart:modes:europe:mode2}

        É necessário um equipamento de controle no conector ou no cabo, sendo que este equipamento fará o sequenciamento elétrico com o veículo (descrito na seção \ref{stateofart:pwm-seq}). Utiliza o mesmo conector do Modo 1 e possui as mesmas limitações elétricas de tensão, porém permite um máximo de até 32 A de corrente. Dentro desse equipamento de controle há também um circuito de proteção, o que permite o uso desse modo em locais não aterrados. É utilizado em alguns locais públicos da Europa e é considerado uma solução de transição nos EUA. Algumas fabricantes disponibilizam um carregador modo 2 para ser utilizado em casa, visto que é somente um cabo com uma caixa de controle. Nesse modo, o veículo precisa possuir com um inversor interno para a conversão \ac{CA}-\ac{CC}.

        \subsubsection{Modo 3 - Carregamento Rápido CA}
        \label{stateofart:modes:europe:mode3}

        O cabo é conectado à uma \textit{\ac{EVSE}}, sendo que essa precisa estar habilitada à realizar \textit{\ac{PWM}} e possuir uma proteção elétrica interna. O carregamento é ajustado de acordo com o sinal recebido via \textit{\ac{PWM}}, sendo que seu sequenciamento é explicado na seção \ref{stateofart:pwm-seq}. Com o uso de 400 V trifásico com 63 A de corrente (especificações máximas deste modo), é possível carregar certos carros em menos de uma hora. Esse modo está se tornando cada vez mais comum, porém - para obter aproveitamento máximo - o veículo precisa possuir um inversor adequado corrente máxima da especificação, o que geralmente não ocorre e acaba limitando a velocidade do carregamento.

        \subsubsection{Modo 4 - Carregamento Rápido CC}
        \label{stateofart:modes:europe:mode4}

        Este modo ultra-rápido permite tensões de 400 V e corrente de 200 A, utilizando \ac{CC} para tal. Como a estação é a responsável pela conversão \ac{CA}-\ac{CC}, esse modo requer que esta possua um inversor, sendo desnecessário o inversor do veículo. Estações que permitem carregamento no modo 4 custam mais caro que estações modo 3, devido aos requisitos de segurança e a necessidade de um bom inversor para atingir potências elevadas. Porém, essas estão se tornando cada vez mais atrativas devido a velocidade de seus carregamentos.

    \subsection{Modos Americanos}
    \label{stateofart:modes:us}

      \begin{itemize}
        \item Nível 1: assim como o modo 1 europeu, utiliza um conector residencial (americano) para o carregamento, fornecendo até 120 V \ac{CA} ao veículo.
        \item Nível 2: pode fornecer 240 ou 208 V \ac{CA} e de 20 à 100 A. Em instalações residenciais, normalmente acaba limitado à 30 A, podendo oferecer até 7,2 kW de potência. Este é o modo mais comum de instalação em residências americanas.
      \end{itemize}

      Há ainda modos de corrente contínua, sendo que esses são idênticos ao Modo 4 Europeu.

    \subsection{Outros modos}
    \label{stateofart:modes:other}

      \subsubsection{Indução}
      \label{stateofart:modes:other:induction}

        O veículo pode ser carregado sem precisar estar conectado à uma estação, o que oferece maior segurança e comodidade para o motorista. Há três maneiras de realizar o carregamento \cite{ieee-review-evse}:

        \begin{itemize}
          \item Estática: veículo é carregado enquanto está estacionado
          \item Quasi-estática: o veículo é carregado enquanto há pessoas dentro, porém em locais específicos (parado no trânsito, por exemplo)
          \item Dinamicamente: enquanto o veículo está em movimento, como em uma rodovia
        \end{itemize}

        Ainda há muita pesquisa nesse modo, principalmente devido a sua baixa eficiência quando comparado aos modos cabeados.

      \subsubsection{Troca de Baterias}
      \label{stateofart:modes:other:swap}

        Há a possibilidade de carregamento por troca de baterias, onde o veículo para em uma estação e um sistema automatizado remove a bateria atual do carro e a substitui por uma carregada \cite{battery-swap}. Essa opção oferece segurança para o motorista e não sofre do problema de eficiência do carregamento por indução. 

      \subsubsection{Tesla Supercharge}
      \label{stateofart:modes:tesla}

        Nos modos cabeados, ainda há o modo proprietário da fabricante Tesla, o \textit{Tesla Supercharge}, que permite o carregamento de 50\% da bateria do Tesla S (até 120kWh) em até 30 minutos \cite{tesla-supercharge}. Para possibilitar tal carregamento, os carregamentos são realizados em corrente contínua, assim como o Modo 4 Europeu.

  \section{Sequenciamento elétrico de carregamentos CA}
  \label{stateofart:pwm-seq}
    Como descrito na seção \ref{stateofart:modes:europe}, os carregamentos modo 2 e 3 requerem um sequenciamento para a inicialização e finalização de seus carregamentos, definidos pela \cite{iec-62196}. Esse sequenciamento é dado por meio de \textit{\ac{PWM}} e variações dos níveis de tensão, o que permite realizar todo controle necessário.

    Na figura \ref{fig:phoenix}, é possível ver um exemplo do sequenciamento de um carregamento gerenciado por dispositivos da Phoenix Contact. No estado A, quando não há um veículo conectado, o conector apresenta uma saída de 12V~-12V constante. Assim que é conectado, a tensão cai para 9V~-12V e o estado muda para B. 

    Enquanto for mantido 9V constante (estado B1), a estação ainda não está pronta para o carregamento, porém assim que começar houver um sinal modulado por \textit{\ac{PWM}} (estado B2), a estação indica que está pronta. A tensão então cai para 6V~-12V ou 3V~-12V (estado C ou D) e o veículo inicia o carregamento. Pelo \textit{duty cicle} do \textit{\ac{PWM}}, a estação pode informar ao veículo quanta corrente pode ser disponibilizada, sendo que durante o carregamento esse \textit{duty cicle} pode ser re-ajustado. 

    Ao termino do carregamento, o veículo desativa o \textit{\ac{PWM}} e a estação volta para o estado B. Após o conector ser removido pelo usuário, a estação vai para o estado A e permanece assim até o próximo carregamento.

    Caso houver uma falha em alguma dessas etapas, o nível de tensão pode ir para 0~-12V ou o sinal é interrompido.

    \begin{figure}[H]
      \begin{center}
        \includegraphics[width=0.75\textwidth,natwidth=400,natheight=288]{assets/images/phoenix.jpg}
        \caption{Exemplo de sequenciamento para carregamentos CA}
        \label{fig:phoenix}
      \end{center}
    \end{figure}


  \section{Padrões de Conectores}
  \label{stateofart:plugs}

      Diversos tipos de conectores estão disponíveis hoje no mercado. Um dos padrões mais aceitos para o carregamento lento é o Tipo 2 - \textit{Mennekes}. Já foi submetido para se tornar o padrão oficial desse tipo de carregamento na Europa \cite{mckinsey-report-ev}.

      Para carregamentos rápidos porém, existem três conectores que são bastante usados: o CHAdeMO, o CCS Combo e o Tesla Supercharger. Normalmente, as estações fornecem o conector \textit{Mennekes} e uma ou duas opções de carregamento rápido, o que é similar ao que ocorre em estações para veículos movidos a combustão (gasolina e álcool no mesmo posto).

      \subsection{Tipo 2 - Mennekes}
      \label{stateofart:plugs:mennekes}

        Proposto pela empresa \textit{Mennekes}, permite carregamentos \ac{CA} monofásicos/trifásicos e \ac{CC} (de baixa e média potências), além de ser retro-compatível com conectores Tipo 1, mesmo possuindo 7 pinos (contra 5 do tipo 1). Permite que a corrente flua de forma bi-direcional, o que permite que os \textit{\ac{EV}} possam fornecer energia para a estação, previsto no modelo de \textit{Vehicle-to-Grid}. É muito utilizado em carregamentos de modo 2 e modo 3, o que o tornou padrão na Europa \cite{mennekes-news-standardplug}, porém para o modo 4 outros tipos são necessários.

        \begin{figure}[H]
          \begin{center}
            \includegraphics[width=0.50\textwidth,natwidth=1024,natheight=755]{assets/images/connectors-mennekes.jpg}
            \caption{Conector Tipo 2 - \textit{Mennekes}}
            \label{fig:mennekes}
          \end{center}
        \end{figure}

      \subsection{Tipo 2 - CCS Combo}
      \label{stateofart:plugs:combo}

        Permite carregamentos rápidos em \ac{CC} e \ac{CA}, sendo uma evolução do Tipo 2 - \textit{Mennekes}. O conector possui o mesmo layout de pinos do \textit{Mennekes}, porém é acrescido de dois pinos extras na sua parte inferior para permitir carregamentos \ac{CC} \cite{ieee-review-evse}.

        \begin{figure}[H]
          \begin{center}
            \includegraphics[width=0.40\textwidth,natwidth=1024,natheight=973]{assets/images/connectors-combo.jpg}
            \caption{Conector Tipo 2 - CCS Combo}
            \label{fig:combo}
          \end{center}
        \end{figure}

      \subsection{CHAdeMO}
      \label{stateofart:plugs:chademo}

        Utilizado em carregamentos modo 3 e 4, foi o primeiro conector que possibilitou carregamentos rápidos e \ac{CC}. Hoje, porém, outros padrões como o CCS Combo (\ref{stateofart:plugs:combo}) estão oferecendo um suporte satisfatório para carregamentos rápidos \cite{ieee-review-evse}. Visto a preocupação da CHAdeMO com segurança e o alto nível de tensão, esse conector possui 10 pinos.

        \begin{figure}[H]
          \begin{center}
            \includegraphics[width=0.35\textwidth,natwidth=300,natheight=300]{assets/images/connectors-chademo.jpg}
            \caption{Conector CHAdeMO}
            \label{fig:chademo}
          \end{center}
        \end{figure}

      \subsection{Tesla Supercharger}
      \label{stateofart:plugs:tesla}

        Fornecido pela \textit{Tesla Motors}, foca em carregamentos rápidos em \ac{CC} - modo 4 - e empresas como Nissan e BMW já negociam um acordo com a Tesla para a utilização do conector em seus veículos, o que permite que sua frota utilize a infra-estrutura de carregamentos \textit{Tesla Supercharge} \cite{ieee-review-evse}.

        \begin{figure}[H]
          \begin{center}
            \includegraphics[width=0.35\textwidth,natwidth=300,natheight=300]{assets/images/connectors-tesla.jpg}
            \caption{Conector Tesla Supercharger}
            \label{fig:tesla}
          \end{center}
        \end{figure}

  \section{Protocolo OCPP}
  \label{stateofart:ocpp}

    As estações de carregamento normalmente se comunicam com um servidor central, que pode gerenciar N estações. Para tal tarefa, é necessário um protocolo de comunicação. Embora ainda não exista um padrão oficial, o \textit{\ac{OCPP}} é um padrão \textit{de facto} e já existem esforços para o tornar um padrão oficial junto a \textit{\ac{OASIS}} \cite{ocpp-news-standardization}.

    Mantido e criado pela \textit{\ac{OCA}}, o \textit{\ac{OCPP}} está presente em mais de 50 países. Na Europa, todas estações comercializadas precisam ser compatíveis com o \textit{\ac{OCPP}} e, na América, o interesse da indústria aumentou nos últimos anos \cite{forbes-news-ocpp}.

    O protocolo prevê um sistema central que recebe dados de N estações. Caso for necessário, o sistema central pode atuar sob \textit{\ac{EVSE}} específicas com ações como reservar a estação, cancelar algum carregamento ou até desligá-la. Caso a estação perca conectividade, o protocolo prevê um modo de funcionamento autônomo, somente registrando alguns dados para envio posterior (início e finalização de carregamentos) \cite{ocpp-spec-15}.

    Hoje existem duas versões: 1.5 e 1.6. Em ambas versões o protocolo funciona como um \textit{web service}, um tipo de serviço \textit{web} que funciona sob o \textit{stack} \textit{TCP/IP} e pode possuir diferentes tipos de implementações, como \textit{WebSocket} e \textit{\ac{SOAP}}. A versão 1.5, por exemplo, pode ser servida por meio de \textit{WebSocket} ou \textit{\ac{SOAP}}, sendo que o segundo é servido sob protocolo \textit{\ac{HTTP}}. Já a versão 1.6 também pode ser servida via \textit{\ac{REST}}.

    As versões 1.6 e 2.0 adicionam novos tipos de requisições, porém é a versão 1.5 que define as requisições base, visto que suas requisições também são suportadas nas versões mais novas. As requisições suportadas pela versão 1.5 são apresentadas na lista a seguir \cite{ocpp-spec-15}:

    \begin{itemize}
      \item \underline{Enviadas pela estação:}
      \begin{itemize}
        \item Authorize: verifica se o usuário está autorizado à usar a estação
        \item BootNotification: notifica a inicialização da estação e checa se ela pode se conectar ao servidor
        \item DataTransfer: requisita variáveis e requisições específicas que não são padronizadas pelo OCPP
        \item DiagnosticStatusNotification: notifica o estado do envio dos dados de diagnósticos da estação
        \item FirmwareStatusNotification: notifica o estado da atualização de \textit{firmware}
        \item Heartbeat: notifica o servidor que a estação ainda está conectada
        \item MeterValues: envia medições dos carregamentos da estação (corrente/tensão/potência...)
        \item StartTransaction: notifica a inicialização de um carregamento
        \item StatusNotification: notifica a mudança de estado de um conector
        \item StopTransaction: notifica a finalização de um carregamento
      \end{itemize}
      \item \underline{Recebidas pela estação:}
      \begin{itemize}
        \item CancelReservation: cancela a reserva do conector atual
        \item ChangeAvailability: modifica a disponibilidade de um conector
        \item ChangeConfiguration: modifica variáveis internas da estação
        \item ClearCache: limpa o cache da estação (como a lista de usuários que foram autorizados)
        \item DataTransfer: recebe requisições específicas que não são padronizadas pelo OCPP
        \item GetConfiguration: requisita as variáveis internas da estação
        \item GetDiagnostics: requisita dados de diagnóstico da estação
        \item GetLocalListVersion: requisita a versão da \textit{whitelist} que está sendo utilizada na estação
        \item RemoteStartTransaction: inicializa um carregamento remotamente
        \item RemoteStopTransaction: finaliza um carregamento remotamente
        \item ReserveNow: reserva um conector da estação
        \item Reset: reinicia a estação
        \item SendLocalList: envia uma nova \textit{whitelist} para a estação
        \item UnlockConnector: destrava um conector (caso esse possua trava mecânica controlada pela estação)
        \item UpdateFirmware: atualiza o \textit{firmware} da estação
      \end{itemize}
    \end{itemize}

    Os dados obtidos e as requisições permitidas abrem espaço para algumas aplicações. Com os MeterValues, StatusNotification, ChangeAvailability, StartTransaction e StopTransaction é possível obter um relatório dos carregamentos que ocorreram, realizar a cobrança destes, visto que as transações sempre estão associadas a um usuário, e até atuar no desligamento de um conector em caso de falhas. Requisições como ReserveNow, CancelReservation, RemoteStartTransaction e RemoteStopTransaction permitem que donos de veículos elétricos, por meios remotos (exemplo: aplicativos de celular), atuem na estação de modo à iniciar o carregamento do seu veículo ou agendar a estação para uso posterior. As outras requisições são mais focadas na administração e manutenção da estação, o que também é importante visto que nem sempre existirá uma equipe local para gerenciá-la.

    Embora o protocolo implemente diversos tipos de requisições e aplicações, os fabricantes não são obrigados a implementar todas. Isso dá flexibilidade no momento do desenvolvimento da estação.